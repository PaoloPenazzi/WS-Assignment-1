\chapter{Tecnologie Impiegate}
\label{chap:tecnologie}

Il progetto ICOS utilizza diverse tecnologie del web semantico per la gestione dei dati e dei metadati relativi alle misurazioni dell'anidride carbonica e di altri gas serra effettuate dalle stazioni di monitoraggio ICOS. Alcune delle principali tecnologie utilizzate sono:

\section{RDF}
\label{section:rdf}
RDF (Resource Description Framework): il progetto ICOS utilizza il formato RDF per rappresentare i dati e i metadati relativi alle misurazioni delle stazioni di monitoraggio. RDF è un formato standard del web semantico utilizzato per rappresentare le informazioni in modo strutturato.

\section{OWL}
\label{section:owl}
OWL (Web Ontology Language): l'ontologia ICOS Carbon Portal utilizza il linguaggio formale OWL per definire i concetti e le relazioni tra i concetti. OWL è un linguaggio standard del web semantico utilizzato per descrivere ontologie.

\section{SPARQL}
\label{section:sparql}
SPARQL (SPARQL Protocol and RDF Query Language): il progetto ICOS utilizza SPARQL per interrogare e recuperare i dati e i metadati delle stazioni di monitoraggio. SPARQL è un linguaggio di interrogazione standard del web semantico utilizzato per recuperare informazioni dalle ontologie RDF.

\section{SKOS}
\label{section:skos}
SKOS (Simple Knowledge Organization System): il progetto ICOS utilizza SKOS per gestire e organizzare i concetti all'interno dell'ontologia ICOS Carbon Portal. SKOS è uno standard del web semantico utilizzato per definire le tassonomie e le thesauri.

\section{Linked Data}
\label{section:linkeddata}
Il progetto ICOS segue il principio di Linked Data, tecnologia moderna
e avanzata nel campo della gestione dati, che consente di distribuire
i dati tramite collegamenti Internet (creazione di link tra le risorse RDF), sui quali l'utente
può semplicemente fare clic per visualizzare e/o scaricare i dati.  Ciò consente di creare una rete di dati interconnessi
e di aumentare la loro interoperabilità. 
Inoltre, rende possibile la \textit{machine-to-machine communication}, 
che rappresenta lo scambio di informazioni (prevalentemente automatico,
quindi senza intervento umano) tra dispositivi di varia natura 
oppure mediante un sistema di elaborazione dati centrale.



