\chapter{Dati utilizzati}
\label{chap:dati}

Il progetto ICOS fornisce dati standardizzati e di altas qualità sui gas a effetto serra.
Tutti i dati sono accessibili attraverso la licenza \textit{Creative Commons Attribution 4.0 International license}.
La licenza Creative Commons 4.0 permette di condividere e utilizzare opere originali in modo più flessibile rispetto
al diritto d'autore tradizionale, ma con alcune limitazioni e requisiti per garantire il rispetto dei diritti dell'autore originale.

\section{FAIR principles}
\label{section:fair}
I dati prodotti da ICOS seguono i cosidetti principi \textbf{FAIR}.
I principi FAIR mirano a fornire all'utente strumenti sufficienti per 
comprendere il significato dei dati prima e dopo averli scaricati. 
I principi FAIR definiscono quindi dei requisiti fondamentali per
la gestione dei dati scientifici, al fine di renderli "FAIR",
ovvero Findable (Rintracciabili), Accessible (Accessibili), Interoperable (Interoperabili) e 
Reusable (Riutilizzabili). A tale scopo, ICOS utilizza la tecnologia dei \textit{Linked-Open Data} \ref{section:linkeddata},
tecnologia moderna e avanzata nel campo della gestione dati. 

I quattro principi FAIR sono i seguenti:

\begin{itemize}
    \item Findable: i dati scientifici devono essere facilmente identificabili e rintracciabili attraverso i metadati appropriati.
    Ciò implica l'utilizzo di identificatori univoci e di descrizioni dettagliate dei dati.
    \item Accessible: i dati scientifici devono essere accessibili in modo aperto e gratuito.
    Ciò implica l'utilizzo di licenze adeguate e la disponibilità di strumenti e tecnologie per accedere ai dati.
    \item Interoperable: i dati scientifici devono essere interoperabili, ovvero strutturati in modo standard
    e condivisibili con altre fonti di dati. Ciò implica l'utilizzo di formati standard, di ontologie e di protocolli
    di comunicazione comuni.
    \item Reusable: i dati scientifici devono essere riutilizzabili
    in modo flessibile e senza restrizioni, anche per scopi diversi da quelli
    per cui sono stati creati. Ciò implica la pubblicazione di dati e metadati completi,
    la documentazione dei processi di acquisizione e la creazione di strutture di dati flessibili.
\end{itemize}


\section{title}