%%%%
% Consiglio la visione dei seguenti tutorial:
% - https://www.youtube.com/watch?v=ihxSUsJB_14
% - https://www.youtube.com/watch?v=XTFWaV55uDo
%%%%
\documentclass[12pt,a4paper,openright,twoside]{book}
\usepackage[utf8]{inputenc}

\newcommand{\thesislang}{italian} % decommentare in caso di tesi in italiano
% \newcommand{\thesislang}{english} % commentare in caso di tesi in italiano
\usepackage{thesis-style}
% version
\newcommand{\versionmajor}{0}
\newcommand{\versionminor}{1}
\newcommand{\versionpatch}{2}
\newcommand{\version}{\versionmajor.\versionminor.\versionpatch}
\typeout{Document version: \version}

\begin{document}
	
\frontmatter

% ! TeX root = thesis-main.tex
\title{Title}
\author{Candidate Name Here}
\date{\today}

\newgeometry{margin=0.8in}
\begin{titlepage}
	\begin{center}
		% \vspace*{0.2cm}
		
		\large
		\textbf{ALMA MATER STUDIORUM -- UNIVERSITÀ DI BOLOGNA \\ CAMPUS DI CESENA}
		\\
		\noindent\hrulefill
		\vspace{0.4cm}
		
		\Large
		Scuola di Ingegneria e Architettura \\
		Corso di Laurea (Magistrale) in Ingegneria e Scienze Informatiche
		
		\Huge
		\vspace{4cm}
		\textbf{
			Title Thesis Here
			\\
			Long Titles Should Be Split
			\\
			On Multiple Lines
		}
		
		\large
		\vspace{1cm}
		Tesi di laurea in 
		\\
		\textsc{(Materia di Riferimento)}
		
		\vspace{5.5cm}
		\begin{minipage}[t]{0.64\textwidth}
			\begin{flushleft}
				\textit{Relatore} 
				\\ 
				(\textbf{Prof.} $\mid$ \textbf{Dott.}) \textbf{Nome Cognome}
				\\
				\vspace{0.4cm}
				\textit{Correlatore} 
				\\
				(\textbf{Ing.})? \textbf{Dott.} \textbf{Nome Cognome}
			\end{flushleft}
		\end{minipage}
		\begin{minipage}[t]{0.34\textwidth}
			\begin{flushright}
				\textit{Candidato} 
				\\ 
				\textbf{Nome Cognome}
			\end{flushright}
		\end{minipage}\\
		
		\vfill
		\noindent\hrulefill
		\vspace{0.3cm}
		\Large
		
		(N-Esima) Sessione di Laurea
		\\
		Anno Accademico 20XX-20YY
	\end{center}
\end{titlepage}
\restoregeometry


% \begin{abstract}	
% Max 2000 characters, strict.
% \end{abstract}

% \begin{dedication} % this is optional
% Optional. Max a few lines.
% \end{dedication}

% \begin{acknowledgements} % this is optional
% Optional. Max 1 page.
% \end{acknowledgements}

%----------------------------------------------------------------------------------------
\tableofcontents   
\listoffigures     % (optional) comment if empty
\lstlistoflistings % (optional) comment if empty
%----------------------------------------------------------------------------------------

\mainmatter

%----------------------------------------------------------------------------------------
\chapter{\introductionname}
\label{chap:introduction}
%----------------------------------------------------------------------------------------

Il sistema ICOS è un sistema europeo creato nel 2007 come parte della ricerca 
ambientale europea, con l'obiettivo di sviluppare una rete di osservazione del carbonio in Europa
in grado di monitorare le emissioni e gli assorbimenti di CO2, nonché di altri gas serra,
in vari ambienti terrestri e marini.\\

ICOS si basa su una vasta rete di stazioni di misura che coprono tutta l'Europa e che forniscono 
dati di alta qualità sul bilancio del carbonio. Queste stazioni di misura sono distribuite in 
tutta Europa in modo da rappresentare una vasta gamma di ambienti, come foreste, prati, paludi,
zone umide, fiumi e oceani. Le stazioni utilizzano tecnologie avanzate per misurare il flusso
di gas serra in queste diverse ambientazioni, come ad esempio la tecnologia di campionamento 
automatico per l'acqua e l'aria, la misurazione del radon e la tecnologia satellitare.\\

ICOS raccoglie questi dati ambientali per valutare l'impatto del cambiamento climatico
sul ciclo del carbonio e per sviluppare strategie per mitigare l'impatto delle attività
umane sul clima. La raccolta e l'analisi dei dati di ICOS sono supportati da tecnologie 
avanzate del web semantico, che consentono la gestione, l'analisi e la condivisione dei dati 
ambientali in modo efficace ed efficiente.\\

ICOS è stato sviluppato in collaborazione con altri sistemi di osservazione globale del carbonio,
come il Global Carbon Project e il Copernicus Climate Change Service.
Il sistema ICOS è stato riconosciuto come una risorsa essenziale per la comprensione
del bilancio del carbonio in Europa e nel mondo, e il suo contributo è stato riconosciuto
dalla comunità scientifica e dalle organizzazioni internazionali, come il Panel Intergovernativo
sui Cambiamenti Climatici (IPCC) e l'Organizzazione per la Cooperazione e lo Sviluppo Economico (OCSE).\\

%----------------------------------------------------------------------------------------
\chapter{Panoramica di ICOS}
\label{chap:panoramica}
%----------------------------------------------------------------------------------------
TODOOOOOOOOOOOOOOOOOOOOOOOOOOOOOOOOOO
%----------------------------------------------------------------------------------------
\chapter{Esempi di risorse}
\label{chap:esempi}
%----------------------------------------------------------------------------------------
TODOOOOOOOOOOOOOOOOOOOOOOOOOOOOOOOOOO
%----------------------------------------------------------------------------------------
\chapter{Relazioni}
\label{chap:relazioni}
%----------------------------------------------------------------------------------------
TODOOOOOOOOOOOOOOOOOOOOOOOOOOOOOOOOOO
%----------------------------------------------------------------------------------------
\chapter{Progetti}
\label{chap:progetti}
%----------------------------------------------------------------------------------------
TODOOOOOOOOOOOOOOOOOOOOOOOOOOOOOOOOOO
%----------------------------------------------------------------------------------------
\chapter{\conclusionsname}
\label{chap:conclusions}
%----------------------------------------------------------------------------------------
TODOOOOOOOOOOOOOOOOOOOOOOOOOOOOOOOOOO

%----------------------------------------------------------------------------------------
% BIBLIOGRAPHY
%----------------------------------------------------------------------------------------

%\nocite{*} % uncomment this to show all the reference in the .bib file
\bibliographystyle{plain}
\bibliography{bibliography}


\end{document}