\chapter{Organizzazioni e Soggetti coinvolti}
\label{chap:soggetti}

\section{Carbon Portal}
\label{section:carbonportal}
Il Carbon Portal è il portale web ufficiale di ICOS 
per l'accesso e la condivisione dei dati e delle
informazioni relative al monitoraggio
del ciclo del carbonio e dei gas serra in Europa.\\

Il Carbon Portal di ICOS offre una vasta gamma di servizi e
strumenti per la visualizzazione e l'analisi dei dati,
tra cui mappe interattive, grafici, tabelle e altri strumenti
di visualizzazione dei dati.
Inoltre, il portale fornisce un accesso unico a tutti i
dati e prodotti ICOS,
nonché a informazioni dettagliate sulle attività di
monitoraggio e sui protocolli di misura utilizzati.\\

Il servizio fornito dal Carbon Portal è hostato dall'Università di Lund
(in Svezia) \cite{LundUniversityICOS}
e dall'Università di Wageningen (in Olanda) \cite{WageningenUniversityICOS}.
Tutti i dati disponibili sono a cura dei Thematic Centres e del CAL.
Il Carbon Portal deve inoltre assicurare tutti gli aspetti legati
che circondano la qualità a 360° del dato tra cui la 
sicurezza di questi, fare in modo che i dati siano user-friendly e
soprattutto machine-friendly. Inoltre, la potenza del
Carbon Portal si può ritrovare nell'essere un univo
punto di accesso integrativo
per tutti gli utenti e le parti interessate ICOS,
che vanno da esperti al grande pubblico e supporta
protocolli e tecniche avanzate di scambio di dati, ampiamente standardizzate.
In particolare, tra le attività e le caratteristiche di tale servizio troviamo:
\begin{itemize}
    \item \textbf{Archiviazione a lungo termine e servizio di backup}.
    Il Carbon Portal deve organizzare l'archiviazione 
    a lungo termine dei dati ICOS
    prodotti, con l'obiettivo di garantirne la sicura
    archiviazione e gli accessi futuri (anche dopo un possibile
    cessazione dell'infrastruttura di ricerca stessa).
    \item \textbf{Data mining e data extraction}.
    Le funzionalità di ricerca consentono agli utenti di
    individuare e recuperare i dati di interesse, ad esempio,
    limitando la ricerca a specifici tipi di variabili,
    aree geografiche o periodi di tempo.
    \item \textbf{Mantenere gli standard sui dati e metadati}.
    L'armonizzazione degli standard di dati e metadati, è 
    responsabilità del Carbon Portal. Gli standard sui metadati,
    (come ad esempio, ISO 19115, Dublin Core, DIF) così come
    l'applicazione dello standard della direttiva INSPIRE
    (Infrastructure for Spatial Information in the European
    Community,vdirettiva dell'Unione Europea che mira
    a creare un'infrastruttura di scambio di informazioni
    geografiche interoperabile a livello europeo).
    \item \textbf{Fornire un servizio web pubblico}.
    Per agevolare molteplici operazioni sui dati, il Carbon
    Portal mette a disposizione un servizio web pubblico, come si
    può notare dalla figura seguente \ref{figure:carbonportalweb}

    \begin{figure}[h!]
        \centering
        \includegraphics[height=0.5\textwidth]{figures/carbonportalweb.JPG}
        \caption{Interfaccia web messa a disposizione dal Carbon Portal.}
        \label{figure:carbonportalweb}
    \end{figure}

    L'interfaccia web fornisce un impressionante numero di informazioni
    e permette all'utente di analizzare i dati in maniera immediata e
    intuitiva grazie anche alla possibilità di creare istantaneamente
    dei grafici sui dati di nostro interesse. La figura \ref{figure:scatterchart} 
    mostra un esempio
    con un grafico a dispersione.

    \begin{figure}[h!]
        \centering
        \includegraphics[height=0.5\textwidth]{figures/icosscatterchart.JPG}
        \caption{Esempio di una grafico a dispersione su dati d'esempio. Si noti la possibilità
        di scegliere il tipo di grafico e tante altre personalizzazioni.}
        \label{figure:scatterchart}
    \end{figure}


    \item \textbf{Interfacciarsi con i più importanti data portals europei e non}.
    Esiste un accordo, con altri centri di ricerca e
    data centers,
    su un formato univoco per
    lo scambio dei metadati così da facilitarne lo stesso.
    Carbon Portal collabora inoltre con queste realtà
    per garantire l'accessibilità reciproca a questi dati.
\end{itemize}

\section{Thematic Centres}
\label{section:thematic}
ICOS ricava informazioni da tre diverse tipologie di ambiente:
osservazioni atmosferiche, ecosistemiche e oceaniche. Ogni
tipologia di dato, viene raccolto dal relativo centro di osservazioni,
specializzato in analisi e manutenzione delle infrastrutture sul proprio
specifico dato.

\begin{itemize}
    \item \textbf{Atmosphere Thematic Centre}.
    ICOS ha istituito una rete di torri e stazioni
    (localizzate sia in alture di montagna sia in vicinanza di centri urbani)
    dove vengono raccolti i dati sulle concentrazioni di gas serra 
    nell'atmosfera. Le stazioni che si trovano lontano dai
    principali agenti inquinanti atmosferici
    (ad esempio nell'Artico o nelle Alpi) meglio
    rappresentano i cambiamenti nella composizione
    dei gas serra, rispetto ai siti più inquinati.
    A causa della loro elevazione e della distanza dalle
    principali fonti di gas serra,
    le stazioni remote sono principalmente esposte
    alle masse d'aria che rappresenteranno le
    condizioni atmosferiche sull'Europa centrale.
    Al contrario, le stazioni situate all'interno o
    in prossimità delle città sono importanti per
    comprendere le emissioni urbane; queste stazioni
    svolgono un ruolo importante nell'affrontare
    l'inquinamento urbano e aiutano a verificare
    i limiti internazionali di gas serra.

I dati raccolti presso le stazioni atmosferiche vengono elaborati automaticamente e la qualità controllata dal Centro tematico Atmosfera. I laboratori analitici centrali forniscono i gas di calibrazione e analizzano i campioni per un ulteriore controllo di qualità e per estendere l'insieme dei parametri osservati.
    \item \textbf{Ocean Thematic Centre}.
    \item \textbf{Ecosystem Thematic Centre}.
\end{itemize}

\section{ICOS Central Analytical Laboratories (CAL)}
\label{section:CAL}
TODOOOOOOOOOOOOOOOOOOOOO