\chapter{\conclusionsname}
\label{chap:conclusions}
In conclusione, l'Integrated Carbon Observation System
(ICOS) rappresenta un importante progetto europeo 
volto a fornire dati di alta qualità e interoperabili
sulla quantità di carbonio presente nell'atmosfera e
negli ecosistemi terrestri.\\

Attraverso l'uso di standard semantici come RDF e OWL,
ICOS è in grado di garantire la precisione,
la coerenza e la comprensibilità dei dati raccolti
e condivisi tra diversi soggetti coinvolti,
come scienziati, policymaker e pubblico in generale.\\

L'implementazione di ICOS è un passo significativo
verso la creazione di un ecosistema digitale
più sostenibile e responsabile, capace di
monitorare e mitigare gli impatti delle attività
umane sul clima globale. Tuttavia, è fondamentale
continuare a investire nella ricerca e nello
sviluppo di tecnologie avanzate per migliorare
l'efficacia e l'efficienza di ICOS e di altri
progetti simili, al fine di garantire un futuro
più verde e più equo per tutti.